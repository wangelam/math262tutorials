\documentclass[a4paper, 11pt]{article}
\usepackage{comment} % enables the use of multi-line comments (\ifx \fi) 
\usepackage{lipsum} %This package just generates Lorem Ipsum filler text. 
\usepackage{fullpage} % changes the margin
\usepackage[a4paper, total={7in, 10in}]{geometry}
\usepackage[fleqn]{amsmath}
\usepackage{amssymb,amsthm}  % assumes amsmath package installed
\newtheorem{theorem}{Theorem}
\newtheorem{corollary}{Corollary}
\usepackage{graphicx}
\usepackage{tikz}
\usetikzlibrary{arrows}
\usepackage{verbatim}
\usepackage[numbered]{mcode}
\usepackage{float}
\usepackage{tikz}
    \usetikzlibrary{shapes,arrows}
    \usetikzlibrary{arrows,calc,positioning}

    \tikzset{
        block/.style = {draw, rectangle,
            minimum height=1cm,
            minimum width=1.5cm},
        input/.style = {coordinate,node distance=1cm},
        output/.style = {coordinate,node distance=4cm},
        arrow/.style={draw, -latex,node distance=2cm},
        pinstyle/.style = {pin edge={latex-, black,node distance=2cm}},
        sum/.style = {draw, circle, node distance=1cm},
    }
\usepackage{xcolor}
\usepackage{mdframed}
\usepackage[shortlabels]{enumitem}
\usepackage{indentfirst}
\usepackage{hyperref}
    
\renewcommand{\thesubsection}{\thesection.\alph{subsection}}

\newenvironment{problem}[2][Problem]
    { \begin{mdframed}[backgroundcolor=gray!20] \textbf{#1 #2} \\}
    {  \end{mdframed}}

% Define solution environment
\newenvironment{solution}
    {\textit{Solution:}}
    {}

\renewcommand{\qed}{\quad\qedsymbol}
%%%%%%%%%%%%%%%%%%%%%%%%%%%%%%%%%%%%%%%%%%%%%%%%%%%%%%%%%%%%%%%%%%%%%%%%%%%%%%%%%%%%%%%%%%%%%%%%%%%%%%%%%%%%%%%%%%%%%%%%%%%%%%%%%%%%%%%%
\begin{document}
%Header-Make sure you update this information!!!!
\noindent
%%%%%%%%%%%%%%%%%%%%%%%%%%%%%%%%%%%%%%%%%%%%%%%%%%%%%%%%%%%%%%%%%%%%%%%%%%%%%%%%%%%%%%%%%%%%%%%%%%%%%%%%%%%%%%%%%%%%%%%%%%%%%%%%%%%%%%%%
\large\textbf{Tutorial 1: Series and Convergence Tests for Series} \\
\normalsize  MATH 262\\
September 13-17, 2021\\
\noindent\rule{7in}{2.8pt}
%%%%%%%%%%%%%%%%%%%%%%%%%%%%%%%%%%%%%%%%%%%%%%%%%%%%%%%%%%%%%%%%%%%%%%%%%%%%%%%%%%%%%%%%%%%%%%%%%%%%%%%%%%%%%%%%%%%%%%%%%%%%%%%%%%%%%%%%
% Problem 1
%%%%%%%%%%%%%%%%%%%%%%%%%%%%%%%%%%%%%%%%%%%%%%%%%%%%%%%%%%%%%%%%%%%%%%%%%%%%%%%%%%%%%%%%%%%%%%%%%%%%%%%%%%%%%%%%%%%%%%%%%%%%%%%%%%%%%%%%
\begin{problem}{1}
Evaluate the limit (if possible):
\begin{align*}
    \  a) \; {a_{n}} &= \frac{5-2n}{3n-7} \; \; {\tiny  (Section \; 9.1, Q14)}\\
    \  b) \; {a_{n}} &= \frac{e^{n} - e^{-n}}{e^{n}+e^{-n}} \; \; {\tiny  (Section \; 9.1, Q19)}\\
    \  c) \; {a_{n}} &= \frac{n}{ln(n+1)} \; \; {\tiny  (Section \; 9.1, Q22)}
 \end{align*}
\end{problem}
\begin{solution}\\
a) Divide the numerator and denominator of the expression for an by the highest
power of n in the denominator, that is, by n.
\\
$\lim_{n\to\infty}  \frac{5-2n}{3n-7} = \lim_{n\to\infty} \frac{\frac{5}{n} - \frac{2n}{n}}{\frac{3n}{n} - \frac{7}{n}} = \frac{0-2}{3-0} = -\frac{2}{3} $
\\
\\
b) Divide numerator and denominator by $e^{n}$.
\\
$\lim_{n\to\infty}  \frac{e^{n} - e^{-n}}{e^{n}+e^{-n}} = \lim_{n\to\infty} \frac{\frac{e^{n}}{n}-\frac{e^{-n}}{n}}{\frac{e^{n}}{n}+\frac{e^{-n}}{n}} 
=\lim_{n\to\infty}\frac{1-e^{-2n}}{1+e^{-2n}}(1) =\lim_{n\to\infty}\frac{1-e^{-2n}}{1+e^{-2n}}(\frac{1-e^{-2n}}{1-e^{-2n}})=1
$
\\
\\
c) Note: Cannot apply L'Hôpital's rule to a sequence.  WIll need to find associated function f(x) with real variables, such as  $f(x) = \frac{x}{ln(x+1)} $
\\
$
lim_{x\to\infty}\frac{x}{ln(x+1)}
=^{H.R} \lim_{x\to\infty} \frac{1}{\frac{1}{x+1}} 
= \lim_{x\to\infty} (x+1) = \infty + 1 = \infty
\\
$
Therefore, for the given sequence, limit does not exist.
\\
\end{solution} 
\noindent\rule{7in}{2.8pt}
\\

%%%%%%%%%%%%%%%%%%%%%%%%%%%%%%%%%%%%%%%%%%%%%%%%%%%%%%%%%%%%%%%%%%%%%%%%%
% Problem 2
%%%%%%%%%%%%%%%%%%%%%%%%%%%%%%%%%%%%%%%%%%%%%%%%%%%%%%%%%%%%%%%%%%%%%%%%%%%%%%%%%%%%%%%%%%%%%%%%%%%%%%%%%%%%%%%%%%%%%%%%%%%%%%%%%%%%%%%%

\begin{problem}{2}
State if infinite series converges or diverges.  If sum converges, find the sum.

\begin{align*}
a)\sum\limits_{n=5}^\infty \frac{1}{(2+\pi)^{2n}}  \;\tiny (Section\; 9.2, Q3)
\end{align*}
\end{problem}
\begin{solution}
\\a) Note: This is a geometric series: \sum_{n=5}^\infty r^n, where \; r = 1/(2+\pi)^2

\\$
\sum\limits_{n=5}^\infty \frac{1}{(2+\pi)^{2n}} 
$
$
= {(2+\pi)^{-10}} + \frac{1}{(2+\pi)^{12}} + \frac{1}{(2+\pi)^{14}} + ...$\\$
= (\frac{1}{(2+\pi)^{10}}) (1 + \frac{1}{(2+\pi)^2} + \frac{1}{(2+\pi)^4} + ...)\\
=\frac{1}{(2+\pi)^{10}} \cdot \frac{1}{1-\frac{1}{(2+\pi)^2}}\\
= \frac{1}{(2+\pi)^8 [(2+\pi)^2 -1]}
$
\\
\end{solution} 
\noindent\rule{7in}{2.8pt}
%%%%%%%%%%%%%%%%%%%%%%%%%%%%%%%%%%%%%%%%%%%%%%%%%%%%%%%%%%%%%%%%%%%%%%%%%
% Problem 3
%%%%%%%%%%%%%%%%%%%%%%%%%%%%%%%%%%%%%%%%%%%%%%%%%%%%%%%%%%%%%%%%%%%%%%%%%%%%%%%%%%%%%%%%%%%%%%%%%%%%%%%%%%%%%%%%%%%%%%%%%%%%%%%%%%%%%%%%
\\
\begin{problem}{3}
When dropped, an elastic ball bounces back up to a height
three-quarters of that from which it fell. If the ball is dropped
from a height of 2 m and allowed to bounce up and down
indefinitely, what is the total distance it travels before coming
to rest? ($Section$ 9.2, Q21)
\\Hint: Think about total distance travelled in up and down direction.
\\
\end{problem}
\begin{solution}
\\
total distance travelled = $2 + 2(\frac{3}{4}) +  2(\frac{3}{4}) + 2(\frac{3}{4})(\frac{3}{4}) + 2(\frac{3}{4})(\frac{3}{4}) + 2(\frac{3}{4})(\frac{3}{4})(\frac{3}{4}) + ...$
\\$=2+2(\frac{3}{4}) + 2(\frac{3}{4})^2 + 2(\frac{3}{4})^3 + ...$
\\$=2+2(3/2)(1+\frac{3}{4} + (\frac{3}{4})^2+...$
\\$=2-\frac{3}{1-\frac{3}{4}}$
\\=14 m
\\
\\
\end{solution} 
\noindent\rule{7in}{2.8pt}
\\
%%%%%%%%%%%%%%%%%%%%%%%%%%%%%%%%%%%%%%%%%%%%%%%%%%%%%%%%%%%%%%%%%%%%%%%%%
% Problem 4
%%%%%%%%%%%%%%%%%%%%%%%%%%%%%%%%%%%%%%%%%%%%%%%%%%%%%%%%%%%%%%%%%%%%%%%%%%%%%%%%%%%%%%%%%%%%%%%%%%%%%%%%%%%%%%%%%%%%%%%%%%%%%%%%%%%%%%%%

\begin{problem}{4}
Determine if series converge or diverges.  State which test were used.
\begin{align*}
a)\sum\limits_{n=0}^\infty \frac{n^{100}2^n}{\sqrt{n!}}
\;\tiny (Section\; 9.3, Q22)
\end{align*}

\begin{align*} 
b) \sum\limits_{n=2}^\infty \frac{1}{n^2-ln(n)}
\;\tiny
\end{align*}
\end{problem}
\begin{solution}
\\a)Use Ratio Test.
$\lim_{n\to\infty}  \frac{(n+1)^{100}2^{n+1}}{\sqrt{(n+1)!}}/ \frac{n^{100}2^n}{\sqrt{n!}}
\\ = 2 \lim_{n\to\infty} (\frac{n+1}{n})^{100} \frac{1}{\sqrt(n+1)} =0 \;$(converges)
\\
\\
b) Use Limit Comparison Test
\\Series $\sum\limits_{n=2}^\infty \frac{1}{n^2-ln(n)}$ is similar to $\frac{1}{n^2}$

\\$\lim_{n\to\infty} \frac{\frac{1}{n^2-ln(n)}}{\frac{1}{n^2}}$ =$\lim_{n\to\infty} \frac{n^2}{n^2-ln(n)}$ =^{H.R} $\lim_{n\to\infty} \frac{2n}{2n-\frac{1}{n}} =^{H.R} $\lim_{n\to\infty} \frac{2}{2-\frac{1}{n^2}}$
\\= $\frac{2}{2+\frac{1}{\infty}}$ = $\frac{2}{2+0}$ = 1
\\Thus, series also converges by LCT.


\end{solution} 
\noindent\rule{7in}{2.8pt}


%%%%%%%%%%%%%%%%%%%%%%%%%%%%%%%%%%%%%%%%%%%%%%%%%%%%%%%%%%%%%%%%%%%%%%%%%
% Problem 5
%%%%%%%%%%%%%%%%%%%%%%%%%%%%%%%%%%%%%%%%%%%%%%%%%%%%%%%%%%%%%%%%%%%%%%%%%%%%%%%%%%%%%%%%%%%%%%%%%%%%%%%%%%%%%%%%%%%%%%%%%%%%%%%%%%%%%%%%

\begin{problem}{5}
Use test listed to evaluate convergence of :$\sum\limits_{n=1}^\infty \frac{2^{n+1}}{n^n}$
\\a) Root Test  ($Section$ 9.3, Q38)          b) Ratio Test ($Section$ 9.3, Q40)
\\

\end{problem}
\begin{solution}
\\a) Root Test
Let $a_n = 2^{n+1} / n^n$
\\Then, $\lim_{n\to\infty}  \sqrt[n]{a_n}$ = $\lim_{n\to\infty} \frac{2 \cdot 2^{1/n}}{n} = 0$
\\Because limit is less than 1, by root test, $\sum\limits_{n=1}^\infty a_n$ converges.
\\
\\b) Ratio Test
Let $a_n = 2^{n+1} / n^n$
\\$\frac{a_{n+1}}{a_n} = (\frac{2^{n+2}}{(n+1)^{n+1}})(\frac{n^{n}}{2^{n+1}})
\\=\frac{2}{(n+1)(\frac{n}{n+1})^n}$
\\=$\frac{1}{n+1} \cdot \frac{1}{(1+\frac{1}{n})^n}
\\= 0 \cdot \frac{1}{e}  = 0$ as n approaches $\infty$.
\\So, converges by Ratio Test
\\
\end{solution} 
\noindent\rule{7in}{2.8pt}
\end{document}